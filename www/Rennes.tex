\documentclass[12pt]{article}
%
%
% Retirez le caractere "%" au debut de la ligne ci--dessous si votre
% editeur de texte utilise des caracteres accentues
\usepackage[utf8]{inputenc}
%
% Retirez le caractere "%" au debut des lignes ci--dessous si vous
% utiisez les symboles et macros de l'AMS
\usepackage{amsmath}
\usepackage{amsfonts}
%
%
\setlength{\textwidth}{16cm}
\setlength{\textheight}{21cm}
\setlength{\hoffset}{-1.4cm}
%
%
\usepackage{graphicx}

 \graphicspath{{figures/}}

\begin{document}

     \def\Var{{\rm Var}\,}

%--------------------------------------------------------------------------

\begin{center}
{\Large
	{\sc  CorReg : Régression sur variables Corrélées\\ et Application à l'industrie Sidérurgique}
}
\bigskip

  Clément Théry$^{1}$ \& Christophe Biernacki$^{2}$ \& Gaétan Loridant$^{3}$
\bigskip

{\it
$^{1}$ ArcelorMittal, Université Lille 1, Inria, CNRS, clement.thery@inria.fr
 
$^{2}$ Université Lille 1, Inria, CNRS, christophe.biernacki@math.univ-lille1.fr

$^{3}$ Etudes Industrielles ArcelorMittal Dunkerque, gaetan.loridant@arcelormittal.com\textbf{}
}
\end{center}
\bigskip

%--------------------------------------------------------------------------

{\bf Résumé.} La régression linéaire suppose en général l'usage de variables explicatives indépendantes. Les variables présentes dans les bases de données d'origine industrielle sont souvent très fortement corrélées (de par le process, diverses lois physiques, etc). Le modèle génératif proposé ici consiste à expliciter les corrélations présentes sous la forme d'une structure de sous-régressions linéaires. La structure est ensuite utilisée pour obtenir un modèle parcimonieux libéré des corrélations, facilement interprétable et compatible avec les méthodes de sélection de variables. La structure de corrélations est déterminée à l'aide d'un algorithme de type MCMC. Un package R (\textsc{CorReg}) permet la mise en oeuvre de cette méthode. 
\smallskip

{\bf Mots-clés.} Régression, corrélations, industrie, sélection de variables, modèles génératifs, SEM (Structural Equation Model), \ldots
\bigskip\bigskip

{\bf Abstract.} Linear regression generally suppose independence between the covariates. Datasets found in industrial context often contains many highly correlated covariates (due to the process, physcial laws, etc). The proposed generative model consists in explicit modeling of the correlations with a structure of sub-regressions between the covariates. This structure is then used to obtain a reduced model with independent covariates, easily interpreted, and compatible with any variable selection method. The structure of correlations is found with an MCMC algorithm. An R package (\textsc{CorReg}) implements this new method.  
\smallskip

{\bf Keywords.} Regression, correlations, industry, variable selection, generative models, Structural Equation Model, \ldots

%--------------------------------------------------------------------------

\section{Décorrélation par modèle génératif}
	La régression linéaire classique suppose l'indépendance des covariables. Les corrélations posent en effet des problèmes, tant au niveau de l'interprétation qu'en termes de variance des estimateurs. La régression $Y=XA+\varepsilon$ avec $ \varepsilon\sim \mathcal{N}(0,\sigma^2)$ donne un estimateur de variance $ \Var(\hat{A}|X)= \sigma^2(X'X)^{-1}$ qui explose si les colonnes de $X$ sont linéairement corrélées.
		
		On suppose le modèle génératif suivant :
	\begin{eqnarray}
	Y_{|X,S}&=&XA+\varepsilon_Y= X^{I_2^c}A_{I_2^c}+X^{I_2}A_{I_2}+\varepsilon_Y \textrm{ avec } \varepsilon_Y \sim \mathcal{N}(0,\sigma_Y^2) \label{MainR}\\
	\forall j \in I_2 : \  X^j_{|X^{I_1^j},S}&=&X^{I_1^j}B_{I_1^j}^j + \varepsilon_{j} \textrm{ avec } \varepsilon_j \sim \mathcal{N}(0,\sigma_j^2) \label{SR}\\
    \forall j \notin I_2 : \ X^j &=& f(\theta_j) \textrm{ mélanges gaussiens orthogonaux à $k_j$ composantes} 	
\end{eqnarray}
Où $B_{I_1^j}^j$ est le vecteur de taille $p_1^j$ des coefficients de la sous-régression en $X^j$, $I_1=\{I_1^1,\dots,I_1^{p}\}$, $I_2=\{j |\sharp I_1^j>0 \}$.
On suppose $I_1\cap I_2=\emptyset$, $i.e.$ Les variables dépendantes dans $X$ n'en expliquent pas d'autres.
On note $p_2= \sharp I_2$, $p_1=(p_1^1,\dots,p_1^{p_2})$ et $I_2^c=\{1,\dots,p\}\setminus I_2$.
	
On a donc rendu explicites les corrélations au sein de $X$ sous la forme d'une structure de sous-régressions linéaires $S=(p_2,I_2,p_1,I_1)$.	
Cette structure est identifiable au sens de la complexité de la structure et des mélanges gaussiens via un critère de type BIC sous certaines conditions simples. Du point de vue de l'interprétation, les cas non identifiables (triviaux) sont équivalents et ne posent pas de problème.

On remarque que (\ref{MainR}) et (\ref{SR}) impliquent :
\begin{eqnarray}
	Y_{|X,S}&=&X^{I_2^c} (A_{I_2^c}+ \sum_{j \in I_2}B^{j}_{I_1}A_{j})+  \sum_{j \in I_2}\varepsilon_{j}A_{j}+\varepsilon_Y \\
					&=& X^{I_2^c}\alpha_{I_2^c}+ \varepsilon_{\alpha}=X\alpha+ \varepsilon_{\alpha} \label{Trueexpl} 			
\end{eqnarray}
\section{Estimateur}
		\textsc{CorReg} réduit la variance de l'estimateur en estimant $Y$ seulement à partir de $X^{I_1}$, sachant (\ref{SR}) et (\ref{Trueexpl}).
			On a ainsi : 
		\begin{equation}
			\hat{\alpha}_{I_2^c} = (X'_{I_2^c} X^{I_2^c})^{-1}X'_{I_2^c}Y \textrm{ et } \hat{\alpha}_{I_2} = 0
		\end{equation}
		estimateur sans biais \cite{saporta2006probabilites} avec :
		\begin{eqnarray}
			\Var[\hat{\alpha}_{I_2^c}|X,S]&=& (\sigma^2_Y+\sum_{j \in I_2}\sigma^2_{j}A_{j}^2 )(X'_{I_2^c} X^{I_2^c})^{-1} \\
		\end{eqnarray}
		La variance est réduite (retrait des corrélations et réduction de la dimension améliorent drastiquement le conditionnement) pour les faibles valeurs de $\sigma_j$ $i.e.$ les fortes corrélations.					
		
		Le modèle complet et le nôtre prédisent tous les deux $Y$ sans biais (vrai modèle). La décorrélation se fait au prix d'un bruit blanc supplémentaire $\sum_{j \in I_2}\varepsilon_{j}A_{j} $ qui est d'autant plus faible que les corrélations sont fortes.
	\\	
	Ce nouveau modèle consiste en une régression linéaire classique qui peut donc bénéficier des outils de sélection de variables au même titre que le modèle complet.			
		 \\		
		 La structure explicite entre les variables permet de mieux comprendre les phénomènes en jeu et la parcimonie du modèle facilite son interprétation.		

		En ajoutant une étape de sélection de variable on obtient deux types de $0$ : les $0$ de corrélation issus de la structure et qui sont à interpréter comme des $0$ de redondance d'information (qui ne signifient donc en rien l'indépendance avec $Y$) et les $0$ de sélection issus de l'éventuelle méthode de sélection de variables (type LASSO) et qui sont à interpréter comme l'indépendance entre la variable explicative concernée et la variable réponse.
			\\
		Le modèle obtenu est donc sans biais de prédiction, parcimonieux et consistant en interprétation.	
\section{Recherche de structure}
	 	
	%	\subsection{Pénalisation de la vraisemblance}
		Le choix de structure s'appuie sur $BIC^*$, vraisemblance pénalisée de la structure à la manière du critère BIC~\cite{BIChuard}, mais en prenant comme loi a priori sur $S$ une loi uniforme hiérarchique $P(S)=P(I_1 | p_1,I_2,p_2)P(p_1|I_2,p_2)P(I_2|p_2)P(p_2)$ plutôt qu'une loi uniforme simple. 
		\begin{eqnarray}
		P(S|X)&\propto &P(X|S)P(S) \\
%		\ln(P(S|X))&=&\ln(P(X|S))+\ln(P(S))+cste \\
		%&=&BIC +\ln(P(S))+cste \\
		BIC^*&=&BIC +\ln(P(S)) \label{Bicstar}
	\end{eqnarray}	
	L'équiprobabilité ainsi supposée des $p_2$ et $p_1^j$ vient pénaliser davantage la compléxité sous l'hypothèse $p_2<\frac{p}{2}$ (qui devient alors une contrainte supplémentaire dans l'algorithme de recherche). 
	On a

	A chaque étape de l'algorithme MCMC, pour $S \in \mathcal{S}$ (ensemble des structures réalisables) on définit un voisinage $\mathcal{V}_{S}$ de $p$ candidats (le package \textsc{CorReg} permet à l'utilisateur de choisir parmi plusieurs types de voisinage).
		
	On fait l'approximation suivante : 
	\begin{equation}
		P(S|X)\approx \exp(BIC^*(S))
	\end{equation}
	On définit alors
	\begin{eqnarray}
			\forall (S,\tilde{S}) \in \mathcal{S}^2 : \mathcal{P}(S,\tilde{S})&=& \frac{1}{p} \sum_{j=1}^p \mathbf{1}_{ \{\tilde{S}\in \mathcal{V}_{S}\} }\frac{\exp(\frac{-1}{2} BIC(\tilde{S}))}{\sum_{S_l\in \mathcal{V}_{S}}\exp(\frac{-1}{2} BIC(S_l))} \\
	\end{eqnarray}
	
		La chaîne de Markov ainsi constituée est ergodique dans un espace d'états finis et possède une unique loi stationnaire.
		Le résultat obtenu est la meilleure structure rencontrée en termes de $BIC^*$ (vraisemblance pénalisée). 
 
L'intialisation peut se faire en utilisant la matrice des corrélations et/ou la méthode du Graphical Lasso\cite{friedman2008sparse}.		
La grande dimension de l'espace parcouru rend préférable  (pour un temps de calcul égal) l'utilisation de multiples chaînes courtes plutôt qu'une seule très longue (permet aussi la parallélisation).

\section{Résultats}	






\begin{tabular}{|c|c|c|c|c|c|c|}
\hline 
$n$ & $p_2$ & $\sharp$ complet & $\sharp$ \textsc{CorReg} &Complet & \textsc{CorReg} & meilleur \\ 
\hline  
30 & 0 & • & • & • &&\\
\hline 
30 & 8 & • & • & • &&\\ 
\hline 
30 & 16 & 22.16 & 18.84 & 65 033.19 & 1 202.52 & 0.69\\ 
\hline 
30 & 24 & 23.05 (3.7) & 17.29 (2.1) & 1 723.67 (9 300) & 987.75 (6 994)& 0.86 (0.34)\\ 
\hline 
50 & 0 & • & • & •&& \\ 
\hline 
50 & 8 & • & • & • &&\\ 
\hline 
50 & 24 & • & • & • && \\ 
\hline 
\end{tabular} 


Les données industrielles sont fortement corrélées de manière naturelle : largeur et poids d'une brame ($\rho=0.905$), température avant et après un outil ($\rho=0.983$), rugosité des deux faces du produit ($\rho=0.919$), Moyenne et maximum d'une courbe ($\rho=0.911$).
Exemples de Sous-régressions obtenues par \textsc{CorReg} ayant interprétation physique :
\begin{itemize}
	\item Moyenne = f (Min , Max , Sigma ) pour des données courbes
	\item Largeur du produit = f (débit de fonte , vitesse de la coulée continue)	\\
Vrai modèle physique (non linéaire) :

	 Largeur = $\frac{\textrm{débit}}{\textrm{vitesse } \times \textrm{ épaisseur}}$ (Mais dans ce cas précis l'épaisseur est constante)
			\end{itemize}
			
			D'autres sous-régressions traduisent des modèles physiques qui régulent le process...
\\

Exemple de régression sur une variable réponse dans le cadre des données réelles :
\begin{figure}[!h]
	\begin{minipage}[c]{.40\linewidth}
			\includegraphics[width=200px]{figures/histcor_auchan.JPG} 
	\end{minipage} \hfill
   \begin{minipage}[c]{.52\linewidth}
		\begin{tabular}{|c|c|c|}
		\hline 
		  & MSE  & Variables retenues  \\ 
		\hline
		LASSO (lars) & 0.80 & 54 \\ 
		\hline 
		CorReg (et lars) & 0.53 & 24  \\ 
		\hline 
		\end{tabular} 
   \end{minipage}
   \caption{résultats obtenus sur données réelles : $n=117$ et $p=168$. l'erreur est réduite d'un tiers alors que la complexité du modèle est divisée par $2,5$.   }
\end{figure}   
	

\section{Conclusion et perspectives}
	\textsc{CorReg} est fonctionnel et disponible. L'outil a d'ores et déjà montré son efficacité sur de vraies problématiques de régression en entreprise.
	La force de \textsc{CorReg} est la grande interprétabilité du modèle proposé, qui est constitué de plusieurs modèles simples (parcimonieux) et facilement accessibles aux non statisticiens (régressions linéaires) tout en luttant efficacement contre les problématiques de corrélations, omniprésentes dans l'industrie.
	On note néanmoins le besoin d'élargir le champ d'application à la gestion des valeurs manquantes, très présentes dans l'industrie. Cet aspect est envisagé sérieusement pour la prochaine version de \textsc{CorReg}.
	

\section*{Bibliographie}
\bibliography{biblio}{}
\bibliographystyle{plain}
à recopier dans le bon ordre comme demandé ci-dessous.

\noindent [1] Auteurs (année), Titre, revue, localisation.

\noindent [2] Achin, M. et Quidont, C. (2000), {\it Théorie des
Catalogues}, Editions du Soleil, Montpellier.

\noindent [3] Noteur, U. N. (2003), Sur l'intér\^et des
résumés, {\it Revue des Organisateurs de Congrès}, 34, 67--89.
\end{document}

