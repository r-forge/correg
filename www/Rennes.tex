\documentclass[12pt]{article}
%
%
% Retirez le caractere "%" au debut de la ligne ci--dessous si votre
% editeur de texte utilise des caracteres accentues
\usepackage[utf8]{inputenc}
%
% Retirez le caractere "%" au debut des lignes ci--dessous si vous
% utiisez les symboles et macros de l'AMS
\usepackage{amsmath}
\usepackage{amsfonts}
%
%
\setlength{\textwidth}{16cm}
\setlength{\textheight}{21cm}
\setlength{\hoffset}{-1.4cm}
%
%
\begin{document}


%--------------------------------------------------------------------------

\begin{center}
{\Large
	{\sc  Regression for correlated variables :\\ Application in steel industry}
}
\bigskip

 Clément Théry $^{1}$
\bigskip

{\it
$^{1}$ ArcelorMittal Dunkerque, Inria Lille, Université de Lille 1, clement.thery@arcelormittal.com
 
}
\end{center}
\bigskip

%--------------------------------------------------------------------------

{\bf Résumé.} La régression linéaire suppose en général l'usage de variables explicatives indépendantes. Les variables présentes dans les bases de données d'origine industrielle sont souvent très fortement corrélées (de par le process, diverses lois physiques, etc). Le modèle génératif proposé consiste à expliciter les corrélations présentes sous la forme d'une de sous-régressions linéaires. La structure est ensuite utilisée pour obtenir un modèle libéré des corrélations, facilement interprétable et compatible avec les méthodes de sélection de variables. La structure de corrélations est déterminée à l'aide d'un algorithme de type MCMC. Un package R (CorReg) permet la mise en oeuvre de cette méthode. 
\smallskip

{\bf Mots-clés.} Régression, corrélations, industrie, sélection de variables, modèles génératifs, SEM (Structural Equation Model) \ldots
\bigskip\bigskip

{\bf Abstract.} Linear regression generally suppose independence between the covariates. Datasets found in industrial context often contains many highly correlated covariates (due to the process, physcial laws, etc). The proposed generative model consists in explicit modeling of the correlations with a structure of sub-regressions between the covariates. This structure is then used to obtain a model with independent covariates, easily interpreted, and compatible with any variable selection method. The structure of correlations is found with an MCMC algorithm. A R package (CorReg) implements this new method.  
\smallskip

{\bf Keywords.} Regression, correlations, industry, variable selection, generative models, Structural Equation Model \ldots

%--------------------------------------------------------------------------

\section{Le contexte}
	La régression linéaire classique suppose l'indépendance des covariables. Les corrélations posent des problèmes.
	\begin{eqnarray}
		Y&=&XA+\varepsilon \ \ \ \varepsilon\sim \mathcal{N}(0,\sigma^2) \\
		Var(\hat{A}|X)&=& \sigma^2(X'X)^{-1} \textrm{ explose si les colonnes de x sont linéairement corrélées}
	\end{eqnarray}
			
	
	
\section{Le modèle génératif}
On dispose de $p$ variables $X$ fortement corrélées pour expliquer une variable réponse $Y$.
On rend explicite les corrélations au sein de $X$ sous la forme d'une structure de sous-régressions linéaires $S=(p_2,I_2,p_1,I_1)$ définie ainsi :
	\begin{eqnarray}
		I_1&=&(I_1^1,\dots,I_1^{p_2}) \textrm{ avec}		\\
		I_1^j &=& \{i |Z_{i,j}=1 \} \textrm{ indices des covariables qui expliquent $X^j$} \\
		I_2&=&\{j |\sharp I_1^j>0 \}  \textrm{ indices des variables dépendantes} \\
		p_2&=& \sharp I_2 \\
		p_1&=&(p_1^1,\dots,p_1^{p_2}) \textrm{ avec }p_1^j=\sharp I_1^j 
	\end{eqnarray}
	On suppose $I_1\cap I_2=\emptyset$, $i.e.$ Les variables dépendentes dans $X$ n'en expliquent pas d'autres. 
	
	On note $I_2^c=\{1,\dots,p\}\setminus I_2$
Then our generative model can be written :
\begin{eqnarray}
	Y_{|X,S}=Y_{|X}&=&XA+\varepsilon_Y= X^{I_2^c}A_{I_2^c}+X^{I_2}A_{I_2}+\varepsilon_Y \textrm{ with } \varepsilon_Y \sim \mathcal{N}(0,\sigma_Y^2) \label{MainR}\\
	\forall j \in I_2 : \  X^j_{|X^{I_1^j},S}&=&X^{I_1^j}B_{I_1^j}^j + \varepsilon_{j} \textrm{ with } \varepsilon_j \sim \mathcal{N}(0,\sigma_j^2) \label{SR}\\
    \forall j \notin I_2 : \ X^j &=& f(\theta_j) \textrm{ free law}	
\end{eqnarray}
Where $B_{I_1^j}^j$ is the $p_1^j$-sized vector of the coefficients of the subregression.

We note that (\ref{MainR}) and (\ref{SR}) also give :
\begin{eqnarray}
	Y&=&X^{I_2^c} (A_{I_2^c}+ \sum_{j \in I_2}B^{j}_{I_1}A_{j})+  \sum_{j \in I_2}\varepsilon_{j}A_{j}+\varepsilon_Y \\
					&=& X^{I_2^c}\tilde{A}_{I_2^c}+ \tilde{\varepsilon}=X\tilde{A}+ \tilde{\varepsilon}\label{Trueexpl} \\
			\textrm{where }		\tilde{A}_{I_2}&=&0 \\
					\tilde{A}_{I_2^c}&=&A_{I_2^c}+ \sum_{j \in I_2}B^{j}_{I_1}A_{j} 
\end{eqnarray}
\section{Estimateur}
	Classical methods like Ordinary Least Squares (OLS) estimate $Y|X$ and obtain (Maximum Likelihood Estimation): 
		\begin{equation}
			\hat A = (X'X)^{-1}X'Y \textrm{ (ill-conditoned matrix to inverse)}
		\end{equation}
		With following properties :
		\begin{eqnarray}
			E[\hat{A}|X]&=&A \\
			Var[\hat{A}|X]&=& \sigma_Y^2(X'X)^{-1}
		\end{eqnarray}				
		And when correlations are strong, the matrix to invert is ill-conditioned and the variance explodes.
 			
		Our idea is to reduce the variance so we explain $Y$ only with $X^{I_1}$ knowing (\ref{SR}) and (\ref{Trueexpl})
			\begin{equation}
				Y= X^{I_2^c}\tilde{A}_{I_2^c}+ \tilde{\varepsilon}\label{explicatif}
			\end{equation}							
		So the new estimator simply is : 
		\begin{eqnarray}
			\hat{\tilde{A}}_{I_2^c} &=& (X'_{I_2^c} X^{I_2^c})^{-1}X'_{I_2^c}Y \\
			\hat{\tilde{A}}_{I_2} &=& 0
		\end{eqnarray}
		and we get the following properties :
		\begin{eqnarray}
			E[\hat{\tilde{A}}|X]&=&\tilde{A} \\
			Var[\hat{\tilde{A}}_{I_2^c}|X]&=& (\sigma^2_Y+\sum_{j \in I_2}\sigma^2_{j}A_{j}^2 )(X'_{I_2^c} X^{I_2^c})^{-1} \\
			Var[\hat{\tilde{A}}_{I_2}|X]&=& 0 
		\end{eqnarray}
		We see that the variance is reduced (no correlations and smaller matrix give better conditioning) for small values of $\sigma_j$ $i.e.$ strong correlations.					
		
		Both classical and our new estimators of $Y$ are unbiased (true model)\cite{saporta2006probabilites}.
	\\	
			This new model is reduced even without variable selection and is just a linear regression so every method for variable selection in linear regression can be used. 
		 \\				
			The explicit structure between the covariates helps to understand the model and the complex link between the covariate and the response variable so we call this model explicative.	
			
			When we use a variable selection method on it we obtain two kinds of 0 :
			\begin{enumerate}
			\item Because of the structure we coerce $\hat{\tilde{A}}^{I_2} = 0 $. This kind of zero means redundant information but the covariate can be correlated with the response variable. So we don't have the grouping effect (so we are more parsimonious ) and we don't suffer from false interpretation (LASSO would).
			\item Variable selection methods can lead to get some exact zeros in $\hat{\tilde{A}}^{I_1}$. This kind of zero means that implied covariate has no significant effect on the response variable. And because variables in $X^{I_1}$ are orthogonal, we know that it is not misleading interpretation due to correlations.
			\end{enumerate}
\section{Recherche de structure}
	 	
	%	\subsection{Pénalisation de la vraisemblance}
		On va s'appuie sur la vraisemblance pénalisée de la structure à la manière du critère BIC~\cite{BIChuard}. 
		\begin{eqnarray}
		P(S|X)&\propto &P(X|S)P(S) \\
		\ln(P(S|X))&=&\ln(P(X|S))+\ln(P(S))+cste \\
		%&=&BIC +\ln(P(S))+cste \\
		BIC^*&=&BIC +\ln(P(S)) \label{Bicstar}
	\end{eqnarray}	
	Pour éviter une surcomplexité de la structure trouvée, on peut alors faire des hypothèses a priori sur $P(S)$. Par exemple, au lieu de supposer l'équiprobabilité pour tous les $S$, on peut supposer l'équiprobabilité des $p_2$ et $p_1^j$, ce qui vient pénaliser davantage la compléxité sous l'hypothèse $p_2<\frac{p}{2}$ (qui devient alors une contrainte supplémentaire dans l'algorithme de recherche). 
	On a
	\begin{eqnarray}
		P(S)&=&P(I_1 | p_1,I_2,p_2)P(p_1|I_2,p_2)P(I_2|p_2)P(p_2) % \\
%		P(I_1 | p_1,I_2,p_2)&=&\prod_{j =1}^{p_2}P(I_1^j|p_1^j,I_2,p_2) \\
%		P(I_1^j|p_1^j,I_2,p_2)&=&\left(\begin{array}{c}
%			p-p_2 \\ 
%			p_1^j
%			\end{array}  \right)^{-1} =\frac{p_1^j ! (p-p_2-p_1^j)!}{(p-p_2)!}\\
%		P(p_1|I_2,p_2)&=&\prod_{j =1}^{p_2}P(p_1^j|I_2,p_2)		\\
%		P(p_1^j|I_2,p_2)&=&\frac{1}{p-p_2}  \\
%		P(I_2|p_2)&=&\left(\begin{array}{c}
%			p \\ 
%			p_2
%			\end{array}  \right)^{-1}=\frac{p_2!(p-p_2)!}{p!}\\
%		P(p_2) &=&\frac{1}{p_2} \\
%		P(S)&=&\left(\prod_{j =1}^{p_2}\left(\begin{array}{c}
%			p-p_2 \\ 
%			p_1^j
%			\end{array}  \right)^{-1}\right) \left(\frac{1}{p-p_2}\right)^{p_2}\frac{p_2!(p-p_2)!}{p!}\frac{1}{p_2} \\
%			\ln P(S) &=& -\sum_{j=1}^{p_2}	\ln\left(\begin{array}{c}
%			p-p_2 \\ 
%			p_1^j
%			\end{array}  \right)
%			-p_2\ln (p-p_2)
%			-\ln\left(\begin{array}{c}
%			p \\ 
%			p_2
%			\end{array}  \right)
%			-\ln( p_2	)
	\end{eqnarray}		
%	It increases penalty on complexity for $p_2<\frac{p}{2}$ thus in the following we will use $BIC*$ under this hypothesis (that becomes a constraint in the MCMC).			
%	\subsection{Parcours Markovien de saut}
	$S$ est entièrement défini à partir de $I_1$ donc on se contente ici de modifier $I_1$. %(les autres éléments de $S$ seront ensuite recalculés selon leur définition).
	A chaque étape, pour $S \in \mathcal{S}$ on définit un voisinage $\mathcal{V}_{S,j}$ avec $j \sim \mathcal{U}(\{1,\dots,p\}) $ :% de la manière suivante  :	
	\begin{eqnarray}
		\mathcal{V}_{S,j}&=&\{ S^{(i,j)} | 1\leq i\leq p \} \cup\{S \}
	\end{eqnarray}	
	avec $S^{(i,j)}$ obtenu selon l'algorithme :
	\begin{itemize}
		\item Si $i \notin I_i^j$ (ajout): 
			\begin{itemize}
				\item $I_1^j :=I_1^j\cup \{i\}$, et pour garder $I_1\cap I_2=\emptyset$ :
				\item $I_1^i :=\emptyset$ et $I_1:=I_1 \setminus \{j\}$
			\end{itemize}			 
		\item Sinon ( $i \in I_1^j$ (suppression)): $I_1^j=I_1^j\setminus \{i\}$
	\end{itemize}
	
	On a donc $p$ candidats à chaque étape. Mais CorReg (notre package) permet à l'utilisateur de modifier ce voisinage.
	
		
	We make a first approximation (\ref{Bicstar}) : 
	\begin{equation}
		P(S|X)\approx exp(BIC^*(S))
	\end{equation}
	\begin{equation}
		q(\tilde{S},\mathcal{V}_{S,j})=\mathbf{1}_{ \{\tilde{S}\in \mathcal{V}_{S,j}\} }\frac{exp(\frac{-1}{2}\Delta BIC(\tilde{S},\mathcal{V}_{S,j}))}{\sum_{S_l\in \mathcal{V}_{S,j}}exp(\frac{-1}{2}\Delta BIC(S_l,\mathcal{V}_{S,j}))}
	\end{equation}
	
	Where $\Delta BIC(S,\mathcal{V}_{S,j})=BIC(S)-\min\{BIC(\tilde{S})| \tilde{S} \in \mathcal{V}_{S,j} \} $.
	\newline
	
	 And then we can note $\forall (S,\tilde{S}) \in \mathcal{S}^2 $ :
		\begin{displaymath}
			\mathcal{P}(S,\tilde{S})= \frac{1}{p} \sum_{j=1}^p q(\tilde{S},\mathcal{V}_{S,j})
		\end{displaymath}
		La chaîne de Markov ainsi constituée est ergodique dans un espace d'états finis et possède une unique loi stationnaire.
		Le résultat obtenu est la meilleur structure rencontrée en termes de $BIC^*$ (vraisemblance pénalisée). Certaines parties de $S$ peuvent être contraintes pour insérer par exemple des modèles physiques. De ce fait, la méthode permet de tenir compte d'éventuels modèles experts.

 
L'intialisation peut se faire en utilisant la matrice des corrélations et/ou la méthode du Graphical Lasso\cite{friedman2008sparse}.		
La grande dimension de l'espace parcouru rend préférable l'utilisation de multiples chaînes courtes plutôt qu'une seule très longue (pour un temps de calcul égal). Accessoirement, les multiples chaînes permettent de paralléliser la recherche, ce qui peut être très appréciable.


	\subsubsection{properties}
	The algorithm follows a time-homogeneous markov chain whose transition matrix $\mathcal{P}$ has $|\mathcal{S}|$ rows and columns (combinatory so we'll just compute the probabilities when we need them).
	And $\mathcal{S}$ is a finite state space.%la relaxation rend P non symétrique mais ne remets  pas en cause l'homogénéité	
	
%	if 	$\forall k, \tilde{Z} \not\in \mathcal{V}_{Z,k} $ then 
We want 
		\begin{equation}
			\mathcal{P}(S,\tilde{S})=\mathbf{1}_{[\exists j, \tilde{S} \in \mathcal{V}_{S,j} ]} P(\tilde{S}|X)
		\end{equation}
			Because the walk follows a regular and thus ergodic markov chain with a finite state space, it has exactly one stationary distribution \cite{grinstead1997introduction} : $\pi$ and every rows of $\operatorname{lim}_{k\rightarrow \infty}\mathcal{P}^k=W$ equals $\pi$.


\section{Résultats}	
\section{Conclusion et perspectives}
	CorReg est fonctionnel et disponible
	Besoin d'élargir à la gestion des valeurs manquantes très présentes dans l'industrie
\section{Exemple de références bibliographiques}
La nécessité de produire des résumés clairs et bien
référencés aété démontrée par Achin et Quidont~(2000). Le
récent article de Noteur~(2003) met enévidence \dots

\section*{Bibliographie}
\bibliography{biblio}{}
\bibliographystyle{plain}
à recopier dans le bon ordre comme demandé ci-dessous.

\noindent [1] Auteurs (année), Titre, revue, localisation.

\noindent [2] Achin, M. et Quidont, C. (2000), {\it Théorie des
Catalogues}, Editions du Soleil, Montpellier.

\noindent [3] Noteur, U. N. (2003), Sur l'intér\^et des
résumés, {\it Revue des Organisateurs de Congrès}, 34, 67--89.
\end{document}

