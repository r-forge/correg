\documentclass[12pt]{article}
%
%
% Retirez le caractere "%" au debut de la ligne ci--dessous si votre
% editeur de texte utilise des caracteres accentues
\usepackage[utf8]{inputenc}
%
% Retirez le caractere "%" au debut des lignes ci--dessous si vous
% utiisez les symboles et macros de l'AMS
\usepackage{amsmath}
\usepackage{amsfonts}
%
%
\setlength{\textwidth}{16cm}
\setlength{\textheight}{21cm}
\setlength{\hoffset}{-1.4cm}
%
%
\usepackage{graphicx}

 \graphicspath{{figures/}}

\begin{document}

     \def\Var{{\rm Var}\,}

%--------------------------------------------------------------------------

\begin{center}
{\Large
	{\sc  CorReg : Régression sur variables Corrélées\\ et Application à l'industrie Sidérurgique}
}
\bigskip

  Clément Théry$^{1}$ \& Christophe Biernacki$^{2}$ \& Gaétan Loridant$^{3}$
\bigskip

{\it
$^{1}$ ArcelorMittal, Université Lille 1, Inria, CNRS, clement.thery@inria.fr
 
$^{2}$ Université Lille 1, Inria, CNRS, christophe.biernacki@math.univ-lille1.fr

$^{3}$ Etudes Industrielles ArcelorMittal Dunkerque, gaetan.loridant@arcelormittal.com\textbf{}
}
\end{center}
\bigskip

%--------------------------------------------------------------------------

{\bf Résumé.} La régression linéaire suppose en général l'usage de variables explicatives décorrélées, hypothèse souvent irréaliste pour les bases de données d'origine industrielle où de nombreuses corrélations sont dues au process, à des lois physiques, {\it etc}. Le modèle  proposé explicite les corrélations présentes sous la forme d'une famille de régressions linéaires entre covariables, permettant d'obtenir par marginalisation un modèle de régression parcimonieux libéré des corrélations, facilement interprétable et compatible avec les méthodes de sélection de variables. La structure de corrélations est estimée à l'aide d'un algorithme de type MCMC. Un package R (\textsc{CorReg}) permet la mise en oeuvre de cette méthode qui sera illustrée sur données simulées et sur données réelles issues de l'industrie sidérurgique.
\smallskip

{\bf Mots-clés.} Régression, corrélations, industrie, sélection de variables, modèles génératifs
\bigskip\bigskip

{\bf Abstract.} Linear regression generally suppose to have decorrelated covariates. This hypothesis is often irrealist with industrial datasets that contains many highly correlated covariates due to the process, physcial laws,  {\it etc}. The proposed generative model consists in explicit modeling of the correlations with a family of linear regressions between the covariates permitting to obtain by marginalization a parsimonious correlation-free regression model, easily understandable and compatible with variable selection methods. The structure of correlations is found with an MCMC algorithm. An R package (\textsc{CorReg}) implements this new method which will be illustrated on both simulated datasets and real-life datasets from steel industry.
\smallskip

{\bf Keywords.} Regression, correlations, industry, variable selection, generative models

%--------------------------------------------------------------------------

\section{Introduction}
	La régression linéaire classique suppose la décorrélation des covariables, source de problèmes en termes de variance des estimateurs. En effet, pour une variable réponse $Y \in \mathcal{R}^{n}$ et un ensemble de covariables $X \in \mathcal{R}^{n \times p}$, la régression $Y=XA+\varepsilon$ avec $ \varepsilon\sim \mathcal{N}(0,\sigma^2I_n)$ (où $I_n$ est la matrice identité de taille $n$) et $A\in \mathcal{R}^{p}$ vecteur des $p$ coefficients donne un estimateur $\hat A$ de variance $ \Var(\hat{A}|X)= \sigma^2_Y(X'X)^{-1}$ dégénéré si les colonnes de $X$ sont linéairement corrélées. Les méthodes de sélection comme le LASSO [4] muni du LAR [1] sont elles-mêmes touchées par ce problème de corrélation [5].
		\\
		
		Notre idée est de modéliser explicitement les corrélations présentes entre covariables sous la forme d'une famille de régressions entre celles-ci. Nous présenterons donc le modèle génératif associé puis en partie \ref{secMCMC} l'algorithme MCMC permettant d'estimer la famille de régressions à utiliser avant d'illustrer dans les parties \ref{secressim} et \ref{secresrel} l'efficacité de la méthode sur des données simulées puis sur des données réelles avant de conclure en partie \ref{secconcl}.
		
\section{Modèle supprimant les covariables corrélées}		
	On suppose le modèle génératif suivant :
	\begin{itemize}
		\item Régression principale entre $Y$ et $X$:
			 \begin{equation}
			 	Y_{|X,S}=XA+\varepsilon_Y= X_1A_{1}+X_2A_{2}+\varepsilon_Y \textrm{ avec } \varepsilon_Y \sim \mathcal{N}(0,\sigma_Y^2I_n); \label{MainR}
			 \end{equation}
		\item Famille de $p_2$ régressions entre covariables de $X$ corrélées :
			\begin{equation}
			\forall j \in I_2 : \  X^j_{|X_1,S}=X_1B_{1}^j + \varepsilon_{j} \textrm{ avec } \varepsilon_j \sim \mathcal{N}(0,\sigma_j^2I_n); \label{SR}
			\end{equation}
		\item Mélanges gaussiens indépendants pour les covariables non corrélées :
			\begin{equation}
				\forall j \notin I_2 : \ X^j \sim \sum_{k=1}^{k_j}\pi_k\mathcal{N}(\mu_{k_j},\sigma_{k_j}^2I_n);
			\end{equation}
	\end{itemize}
	où %$B_{I_1^j}^j$ est le vecteur de taille $p_1^j$ des coefficients de la sous-régression en $X^j$,\\ 
$I_1=\{I_1^1,\dots,I_1^{p}\}$ est le vecteur des indices des variables à droite dans (\ref{SR}) ,  $I_2=\{j |\sharp I_1^j>0 \}$ est l'ensemble des indices des variables corrélées à gauche dans (\ref{SR}). Les $B_1^j \in \mathcal{R}^{(p-p_2)}$ sont les coefficients des régressions entre covariables. On a donc une partition des données $X=(X_1,X_2)$ où $X_2=X^{I_2}$ et $X_1=X\setminus X_2$.
On suppose en outre $I_1\cap I_2=\emptyset$, $i.e.$ les variables dépendantes dans $X$ n'en expliquent pas d'autres.
On note $p_2= \sharp I_2$ le nombre de régressions entre covariables et $p_1=(p_1^1,\dots,p_1^{p_2})$ qui est le vecteur des longueurs des régressions au sein de $X$.
	
On a ainsi rendu explicites les corrélations au sein de $X$ sous la forme d'une structure de sous-régressions linéaires $S=(I_1,I_2,p_1,p_2)$.	
Ce modèle génératif est identifiable sous certaines conditions simples (sur les $k_j$) non détaillées ici.

On remarque alors que (\ref{MainR}) et (\ref{SR}) impliquent par simple intégration sur $X_2$, un modèle de régression en $Y$ s'exprimant {\it uniquement en fonction des variables non corrélées $X_1$} :
\begin{eqnarray}
	Y_{|X_1,S}&=&X_1 (A_{1}+ \sum_{j \in I_2}B^{j}_{I_1}A_{j})+  \sum_{j \in I_2}\varepsilon_{j}A_{j}+\varepsilon_Y 
					= X_1\alpha_{1}+ \varepsilon_{\alpha}. \label{Trueexpl} 			
\end{eqnarray}		
L'estimateur classique du Maximum de Vraisemblance de $\alpha$ est sans biais et s'exprime par	
\begin{equation}
	\hat{\alpha}_{1}= (X'_{1} X_1)^{-1}X'_{1}Y  
\end{equation}
En particulier sa matrice de variance 
\begin{equation}
	\Var[\hat{\alpha}_{1}|X,S]= (\sigma^2_Y+\sum_{j \in I_2}\sigma^2_{j}A_{j}^2 )(X'_{1} X_{1})^{-1}
\end{equation}
		peut être notablement mieux conditionnée que celle de $\hat A$ initial (dimension réduite et surtout variables orthogonales).						
	En outre, ce nouveau modèle réduit consiste en une régression linéaire classique qui peut donc bénéficier des outils de sélection de variables au même titre que le modèle complet.			
		 Notons enfin que la structure explicite entre les variables permet de mieux comprendre les phénomènes en jeu et la parcimonie du modèle facilite son interprétation. 
\\ 
		\textbf{\underline{Remarque}} : En ajoutant une étape de sélection de variables (de type LASSO) on obtient ainsi deux  ``types de $0$"  : ceux issus de l'étape de décorrélation et ceux issus de la sélection.
		
\section{Estimation de la structure de corrélation}\label{secMCMC}
	 	
	%	\subsection{Pénalisation de la vraisemblance}
	
		Le choix de structure s'appuie sur un critère noté $BIC^*$ et qui correspond à la vraisemblance pénalisée de la structure à la manière du critère BIC [3], mais en prenant comme loi {\it a priori} sur $S$ une loi uniforme hiérarchique $P(S)=P(I_1 | p_1,I_2,p_2)P(p_1|I_2,p_2)P(I_2|p_2)P(p_2)$ plutôt qu'une loi uniforme simple. On a donc :
		\begin{eqnarray}
		%P(S|X)&\propto &P(X|S)P(S) \\
%		\ln(P(S|X))&=&\ln(P(X|S))+\ln(P(S))+cste \\
		%&=&BIC +\ln(P(S))+cste \\
		BIC^*&=&BIC +\ln(P(S)) \label{Bicstar} .
	\end{eqnarray}	
	L'équiprobabilité ainsi supposée des $p_2$ et $p_1^j$ vient pénaliser davantage la compléxité sous l'hypothèse $p_2<\frac{p}{2}$ , hypothèse réaliste sur le nombre de régressions entre covariables. La recherche du meilleur $S$ selon $BIC^*$ n'est pas un problème simple et on va s'appuyer sur un algorithme MCMC pour le résoudre.

	A chaque étape de l'algorithme, pour $S \in \mathcal{S}$ (ensemble des structures réalisables) on définit un voisinage $\mathcal{V}_{S}$ %de $p$ candidats (le package \textsc{CorReg} permet à l'utilisateur de choisir parmi plusieurs types de voisinage).
		et ensuite la fonction de transition est guidée par $BIC^*$ de la façon suivante :	
	\begin{eqnarray}
			\forall (S,\tilde{S}) \in \mathcal{S}^2 : P(S,\tilde{S})&=& \mathbf{1}_{ \{\tilde{S}\in \mathcal{V}_{S}\} }\frac{\exp(-\frac{1}{2} BIC^*(\tilde{S}))}{\sum_{S_l\in \mathcal{V}_{S}}\exp(-\frac{1}{2} BIC^*(S_l))}.
	\end{eqnarray}
La chaîne de Markov ainsi constituée est ergodique dans un espace d'états finis et possède une unique loi stationnaire dont le mode correspond à la structure de plus grande valeur de $BIC^*$.
 
L'intialisation peut se faire en utilisant la matrice des corrélations et/ou la méthode du Graphical Lasso [2].		
La grande dimension de l'espace parcouru rend préférable [8]  (pour un temps de calcul égal) l'utilisation de multiples chaînes courtes  plutôt qu'une seule très longue (permettant aussi la parallélisation).
	
	En pratique, on commence par estimer pour chaque variable de $X$ sa densité sous l'hypothèse d'un mélange gaussien (avec le package Rmixmod de Mixmod [6]). On peut alors calculer la loi jointe de $X$ pour chaque structure réalisable rencontrée durant l'algorithme MCMC. Sans cette hypothèse générative supplémentaire sur $X_1$, l'utilisation de $BIC^*$ serait compromise. Notons cependant la souplesse de cette hypothèse due à la grande flexibilité des mélanges gaussiens [7].
\section{Résultats sur données simulées}\label{secressim}	
L'ensemble de la méthode à été programmé dans un package R dénommé \textsc{CorReg}. Pour les simulations présentées dans les tableaux %\ref{tableMSEsimdroite},
 \ref{tableMSEsimtout} et \ref{tableMSEsimgauche}, chacune des configurations à été simulée $100$ fois. Les tableaux affichent le nombre de variables dépendantes trouvées (``bon gauche"), le nombre de variables jugées dépendantes à tort (``faux gauche") et les erreurs moyennes en prédiction (MSE) sur $Y$ à partir d'échantillons de validation de 1 000 individus. Pour l'ensemble des simulations $p=40$, $\sigma_Y=10$, $\sigma=0.001$, les $X$ indépendants suivent des mélanges gaussiens à $\lambda=5$ classes de moyenne selon une loi de Poisson de paramètre $\lambda$ et d'écart-type $\lambda$. Les $B_{1}^j$ suivent la même loi de Poisson mais avec un signe aléatoire. On cherche ici à se comparer à la méthode LASSO dans les cas où celle-ci est en difficulté le vrai modèle est constitué de corrélations 2 à 2. \textsc{CorReg}  a  travaillé avec $p_2$ et $p_1$ libres et a utilisé Rmixmod pour estimer les densités dans $X_1$. 
%
%\begin{table}
%\centering
%\begin{tabular}{|c|c|c|c|c|c|c|c|}
%\cline{3-7}
% \multicolumn{2}{c|}{}  & \multicolumn{2}{c|}{Qualité de $\hat S$} & \multicolumn{3}{c|}{Qualité de prédiction (MSE)} \\
%\hline 
% \multicolumn{2}{c}{}  & \multicolumn{2}{|c |}{Qualité de $S$} & \multicolumn{3}{c|}{Qualité de prédiction (MSE)} \\
%\hline 
%$n$ & $p_2$ & bon gauche & faux gauche    & LAR  &    \textsc{CorReg} $\hat S$& \textsc{CorReg} vrai $S$\\ 
%\hline 
%30 & 16 &  8 & 5.39 & 2 466 225.35 & 13 796.03 & 588.9\\ 
%\hline 
%30 & 32 & 17.05 & 2.7 & 979.16 & 196.33 & 141.2\\ 
%\hline 
%\hline 
%50 & 0 & 0 & 0 & 499.18 & 499.18 & 499.18 \\ 
%\hline 
%50 & 16 & 9.18 & 4.94 & 315.34 & 202.64 & 193.38 \\ 
%\hline 
%50 & 32 & 19.13 & 2.24 & 179.89 & 138.96 & 120.21 \\ 
%\hline \hline
%400 & 32 & 23.66 & 1.13 & 105.38 & 103.88 & 102.81\\ 
%\hline 
%\end{tabular} 
%\caption{$Y$ ne dépend pas de $X_2$.} \label{tableMSEsimdroite}
%\end{table}

\begin{table}
\centering
\begin{tabular}{|c|c|c|c|c|c|c|c|}
\cline{3-7}
 \multicolumn{2}{c|}{}  & \multicolumn{2}{c|}{Qualité de $\hat S$} & \multicolumn{3}{c|}{Qualité de prédiction (MSE)} \\
\hline 
$n$ & $p_2$ & bon gauche & faux gauche    & LAR  &    \textsc{CorReg} $\hat S$& \textsc{CorReg} vrai $S$\\ 
\hline 
30 & 16 & 8.48 & 4.88 & 3 511 185.23 & 10 686.62 & 738.89 \\ 
\hline 
30 & 32 & 16.89 & 2.78 & 565.51 & 189.54 & 139.24\\ 
\hline 
\hline 
50 & 0 & 0 & 0 & 529.94 & 529.94 & 529.94 \\ 
\hline 
50 & 16 & 8.89 & 5.4 & 347.59 & 233.99 & 197.95\\ 
\hline 
50 & 32 & 18.95 & 2.44 & 163.7 & 139.39 & 121.56 \\ 
\hline \hline
400 & 32 & 23.49 & 1.06 & 104.52 & 103.6 & 102.67\\ 
\hline 
\end{tabular} 
\caption{$Y$  dépend  de $X$ entier. \textsc{CorReg} gagne logiquement.} \label{tableMSEsimtout}
\end{table}


\begin{table}
\begin{tabular}{|c|c|c|c|c|c|c|c|}
\cline{3-7}
 \multicolumn{2}{c|}{}  & \multicolumn{2}{c|}{Qualité de $\hat S$} & \multicolumn{3}{c|}{Qualité de prédiction (MSE)} \\
\hline 
$n$ & $p_2$ & bon gauche & faux gauche    & LAR  &    \textsc{CorReg} $\hat S$ & \textsc{CorReg} vrai $S$\\ 
\hline 
30 & 16 & 8.29 & 5 & 5 851.45 & 559.58 & 340.29\\ 
\hline 
30 & 32 & 17 & 2.59 & 893 & 196.01 & 135.78 \\ 
\hline 
\hline 
50 & 16 &  8.98 & 5.19 & 201.56 & 164.58 & 162.49\\ 
\hline 
50 & 32 & 19.05 & 2.32 & 172.93 & 136.77 & 121.19\\ 
\hline \hline
400 & 32 & 23.51 & 1.09 & 104.49 & 103.02 & 102.26 \\ 
\hline 
\end{tabular} 
\caption{$Y$  dépend  de $X_2$ uniquement (cas normalement défavorable à \textsc{CorReg}). } \label{tableMSEsimgauche}
\end{table}


Les tableaux 1 et 2 montrent que \textsc{CorReg} est équivalent au LASSO en l'absence de corrélations et le bat quand les corrélations sont fortes. On retrouve le phénomène attendu du LASSO moins impacté par les corrélations quand $n$ grandit. On constate enfin la convergence asymptotique de \textsc{CorReg} vers le vrai modèle de régression.% en $Y$, comme pour le LASSO.

On remarque que quand $p_2$ augmente le LASSO commence à se ressaisir car il y a de plus en plus de faux modèles proches du vrai en termes de prédiction donc le LASSO trouve des modèles inconsistants en interprétation mais relativement corrects en prédiction. %ATTENTION : dans le cadre industriel, l'interprétation du modèle mène à des actions sur le process et donc un modèle inconsistant en interprétation peut mener à des actions contre-productives (d'où l'importance de l'interprétabilité de \textsc{CorReg}). 
\section{Résultats d'une étude qualité chez ArcelorMittal}\label{secresrel}
%%Exemple de régression sur une variable réponse dans le cadre de données réelles :
\begin{figure}[h!]
	\begin{minipage}[l]{.30\linewidth}
			\includegraphics[height=150px,width=150px]{figures/correlXX1.png} 
			\caption{Valeurs de $\rho$ pour $X$ (haut) et $X_1$ (bas).}
	\end{minipage} \hfill
	\begin{minipage}[c]{.30\linewidth}
			\includegraphics[height=150px, width=150px]{figures/histR2exfos.png} 
			\caption{$R^2_{adj}$ des 82 régressions obtenues.}
	\end{minipage} \hfill
   \begin{minipage}[r]{.30\linewidth}
			\includegraphics[height=150px,width=150px]{figures/p1exfos.png} 
			\caption{Longueur des régressions obtenues ($p_1$).}
   \end{minipage}
\end{figure}   	
On note $\rho$ la valeur absolue des corrélations. Les données sidérurgiques étudiées ($p$=205 et $n$=3000) sont fortement corrélées de manière naturelle (Figure 1 en haut), comme la largeur et poids d'une brame ($\rho$=0.905), la température avant et après un outil ($\rho=0.983$), la rugosité des deux faces du produit ($\rho$=0.919), une moyenne et un maximum ($\rho$=0.911).
%Certaines régressions obtenues décrivent des règles process, d'autres sont plus évidentes :
%\begin{itemize}
%	\item Moyenne = f (Min , Max , Sigma ) pour des données courbes
%	\item Largeur du produit = f (débit de fonte , vitesse de la coulée continue)	\\
%Vrai modèle physique (non linéaire) :
%
%	 Largeur = $\frac{\textrm{débit}}{\textrm{vitesse } \times \textrm{ épaisseur}}$ (Mais dans ce cas précis l'épaisseur est constante)
%			\end{itemize}
\textsc{CorReg} trouve  en plus des corrélations ci-dessus des modèles de régulation du process et des modèles physiques naturels pour un total de $p_2=82$ régressions (Figure 2) de longueur moyenne $\bar p_1=5$ (Figure 3). 
%Les valeurs absolues des corrélations dans $X_1$ sont $5,45\%$ plus faibles que dans $X$, ce qui est significatif vue la dimension de la matrice des corrélations et sa moyenne très faible malgré de fortes corrélations. 
Entre $X$ et $X_1$ le nombre de $\rho > 0.7$ est réduit de $\mathbf{79,33\%}$  avec respectivement $150$ et $31$ paires de variables (Figure 1 en bas).
\\
		Ici $Y$ est un indicateur qualité produit (confidentiel). Le MSE (sur échantillon de validation de 847 nouveaux individus) obtenu par \textsc{CorReg} est $\mathbf{1.55\%}$ meilleur que celui du LASSO, avec respectivement 31 et 20 variables dont 13 communes. LASSO propose 7 variables différentes de  \textsc{CorReg} mais elles sont toutes dans $X_2$ et \textsc{CorReg} reprend les variables explicatives des régressions correspondantes ($R^2_{adj}$ moyen de $0.82$). De plus $\rho$ est $13.9\%$ plus faible pour les variables de \textsc{CorReg} que pour LASSO malgré davantage de variables.
		\\
		En termes d'interprétation, accompagner la régression en $Y$ avec la famille de régressions permet de mieux comprendre les conséquences d'éventuelles mesures correctives sur l'ensemble du process. Cela permet typiquement de déterminer les {\it actionneurs } qui influent sur $Y$ quand le LASSO fait ressortir des variables {\it subies}. On peut donc plus facilement corriger le process pour atteindre l'objectif fixé.
		L'enjeu de ces quelques pourcents de gain se chiffre en dizaine de milliers d'euros annuels sans compter l'impact sur les parts de marché (non chiffrable mais bien plus considérable).
	


\section{Conclusion et perspectives}\label{secconcl}
	\textsc{CorReg} est disponible sur R-forge et a d'ores et déjà montré son efficacité sur de vraies problématiques de régression en entreprise. Sa force est la grande interprétabilité du modèle proposé, qui est constitué de plusieurs régression linéaires courtes et donc facilement accessibles aux non statisticiens tout en luttant efficacement contre les problématiques de corrélations, omniprésentes dans l'industrie.
	On note néanmoins le besoin d'élargir le champ d'application à la gestion des valeurs manquantes, aussi très présentes dans l'industrie. D'ailleurs le modèle génératif actuel permettrait cette nouvelle fonctionnalité sans hypothèse supplémentaire, ce qui renforce encore son intérêt. Enfin, le principe de \textsc{CorReg} qui est l'explicitation des régressions latentes entre covariables pourrait être appliqué à d'autres méthodes prédictives (régression logistique, {\it etc}.). %La connaissance de la structure interne des données possède en effet une valeur intrinsèque que d'autres méthodes pourraient tenter de mettre à profit.
	

\section*{Bibliographie}
%\bibliography{biblio}{}
%\bibliographystyle{plain}
%à recopier dans le bon ordre comme demandé ci-dessous.

\noindent [1] Efron, B., Hastie, T., Johnstone,I. et Tibshirani, R. (2004), Least angle regression. {\it The
Annals of statistics}, 32(2):407-499.

\noindent [2] Friedman, J., Hastie, T. et Tibshirani, R. (2008), Sparse inverse covariance estimation with
the graphical lasso.  {\it Biostatistics}, 9(3):432-441 .

\noindent [3] Lebarbier, E. et Mary-Huard,T. (2006), Une introduction au critère bic: fondements
théoriques et interprétation.  {\it Journal de la SFdS }, 147(1):39-57.

\noindent[4] Tibshirani, R. (1996). Regression shrinkage and selection via the lasso,  {\it Journal of the Royal
Statistical Society}. Series B (Methodological), pages 267-288.

\noindent [5] Zhao,P. et Yu,B. (2006), On model selection consistency of lasso, {\it J. Mach. Learn.
Res.} 7:2541-2563.

\noindent [6] Biernacki, C., Celeux, G., Govaert, G., et Langrognet, F. (2006), Model-based cluster and discriminant analysis with the MIXMOD software, Computational Statistics \& Data Analysis, 51(2), 587-600.

\noindent [7] McLachlan, G., et Peel, D. (2004). Finite mixture models. Wiley. com.

\noindent [8] Gilks, W. R., Richardson, S., et Spiegelhalter, D. J. (Eds.). (1996). Markov chain Monte Carlo in practice (Vol. 2). CRC press.
%
%\noindent [1] Auteurs (année), Titre, revue, localisation.
%
%\noindent [2] Achin, M. et Quidont, C. (2000), {\it Théorie des
%Catalogues}, Editions du Soleil, Montpellier.
%
%\noindent [3] Noteur, U. N. (2003), Sur l'intér\^et des
%résumés, {\it Revue des Organisateurs de Congrès}, 34, 67--89.
\end{document}

