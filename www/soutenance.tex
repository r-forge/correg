\documentclass[11pt]{beamer}
\usetheme{Malmoe}
\usepackage[utf8]{inputenc}
\usepackage[T1]{fontenc}
\usepackage{amsmath}
\usepackage{amsfonts}
\usepackage{amssymb}
\usepackage{graphicx}
  %\usepackage{color}
  \usepackage[utf8]{inputenc}
\usepackage[T1]{fontenc} 
\usepackage[english]{babel}
\usepackage{listings} 
\usepackage[toc,page]{appendix}
\usepackage{array}%,multirow,makecell}
\usepackage{makecell}
\usepackage{amssymb}
\usepackage{subfigure}
\usepackage{placeins}
\usepackage{graphicx}
%\usepackage{framed}
%\usepackage[tikz]{bclogo}
\usepackage{hyperref}
\usepackage{url}
\usepackage{lmodern}
 % \usepackage[usenames,dvipsnames,svgnames,table]{xcolor}%[usenames,dvipsnames,svgnames,table]

\let\oldtabular=\tabular
\def\tabular{\small\oldtabular}
\setcounter{tocdepth}{1}
%  \usetheme{Warsaw}
 \graphicspath{{figures/}}
  \author{Clément Théry, Christophe Biernacki, Gaétan Loridant}\institute{ArcelorMittal Dunkerque, Université de Lille 1,équipe M$\Theta$dal Inria}
\title{CorReg}
%\setbeamercovered{transparent} 
%\setbeamertemplate{navigation symbols}{} 
%\logo{} 
%\institute{} 
%\date{} 
%\subject{} 

%%%% debut macro %%%%
\newenvironment{changemargin}[2]{\begin{list}{}{%
\setlength{\topsep}{0pt}%
\setlength{\leftmargin}{0pt}%
\setlength{\rightmargin}{0pt}%
\setlength{\listparindent}{\parindent}%
\setlength{\itemindent}{\parindent}%
\setlength{\parsep}{0pt plus 1pt}%
\addtolength{\leftmargin}{#1}%
\addtolength{\rightmargin}{#2}%
}\item }{\end{list}}
%%%% fin macro %%%%

\begin{document}

\begin{frame}
\titlepage
\end{frame}

\begin{frame}
\tableofcontents
\end{frame}

\section{Context}
	\subsection{Industrial context}
		\begin{frame}
				  \begin{enumerate}
				\item Steel industry databases.
				\item Goal: To understand and prevent quality problems on finished product, knowing the whole process, \underline{without a priori}.
			\end{enumerate}
	        \begin{center}
	          \begin{tabular}{ccc}
	         \includegraphics[width=70px,height=70px]{liquid.jpg} & \includegraphics[width=70px,height=70px]{Brame1.jpg} & \includegraphics[width=70px,height=70px]{Brame.jpg} \\
	          	\includegraphics[width=70px,height=70px]{ecras_moy.jpg} & \includegraphics[width=70px,height=70px]{tcc2.jpg} & \includegraphics[width=70px,height=70px]{bobines.jpg}
	          \end{tabular}
	        \end{center}
		\end{frame}
	\subsection{Statistical context}
		\begin{frame}{Regression}
			\begin{equation}
				\boldsymbol{Y}=\boldsymbol{X}\boldsymbol{\beta} + \boldsymbol{\varepsilon}\label{regressionsimple}
			\end{equation}
			where 	$\boldsymbol{\varepsilon}\sim \mathcal{N}(0,\sigma_Y^2)$	
		\end{frame}
		
		\begin{frame}{OLS}
		\begin{figure}[h!]
	\centering
	\includegraphics[width=220px]{figures/OLS_geometric_interpretation.png}
	\caption{Multiple linear regression with Ordinary Least Squares seen as a projection on the $d-$dimensional hyperplane spanned by the regressors $\boldsymbol{X}$. Public domain image.} \label{geomOLS}
	\end{figure}
		\end{frame}
		
		\begin{frame}{OLS}
		$\boldsymbol{\beta}$ can be estimated by $\hat{\boldsymbol{\beta}}$ with Ordinary Least Squares (\textsc{ols}), that is the unbiased maximum likelihood estimator \cite{saporta2006probabilites,dodge2004analyse}: %As shown in section \ref{sectionOLS}, 
	\begin{equation}
		\boldsymbol{\hat{\beta}}_{OLS}=\left(\boldsymbol{X}'\boldsymbol{X} \right) ^{-1}\boldsymbol{X}'\boldsymbol{Y}\label{betaOLS}
	\end{equation}
	with variance matrix
	\begin{equation}
		\operatorname{Var}(\hat{\boldsymbol{\beta}}_{OLS})=\sigma_Y^2\left(\boldsymbol{X}'\boldsymbol{X} \right) ^{-1}. \label{eq:varOLS}
	\end{equation}
	 In fact it is the Best Linear Unbiased Estimator (BLUE).
	 The theoretical \textsc{mse} is given by
	\begin{equation}
	\textsc{mse}(\hat{\boldsymbol{\beta}}_{OLS})= \sigma_Y^2 \operatorname{Tr}((\boldsymbol{X}'\boldsymbol{X})^{-1}). \nonumber 
	\end{equation}
		\end{frame}
		
\begin{frame}{Running example}

%$d=5$ with four independent scaled Gaussian 
$\boldsymbol{X}^1,\boldsymbol{X}^2,\boldsymbol{X}^4,\boldsymbol{X}^5  \sim \mathcal{N}(0,1)$  and 
$\boldsymbol{X}^3=\boldsymbol{X}^1+\boldsymbol{X}^2+\boldsymbol{\varepsilon}_1$ where 
$\boldsymbol{\varepsilon}_1\sim{\mathcal{N}(\boldsymbol{0},\sigma_1^2\boldsymbol{I}_n)}$. \\
Two {\it scenarii} for $\boldsymbol{Y}$:\\
 $\boldsymbol{\beta}=(1,1,1,1,1)'$ and $\sigma_Y \in \{10,20\}$.
 \\
It is clear that $\boldsymbol{X}'\boldsymbol{X}$ will become more ill-conditioned as $\sigma_1$ gets smaller. $R^2$ stands for the coefficient of determination which is here:
	\begin{equation}\label{defR2}
	R^2=1-\frac{\operatorname{Var}(\boldsymbol{\varepsilon}_1)}{\operatorname{Var}(\boldsymbol{X}^3)}
	\end{equation}

\end{frame}	
	
	\begin{frame}
	\begin{figure}
	 \centering
	  \includegraphics[width=250px]{figures/MQEreel/OLScompl.png}
	  \caption{Evolution of observed Mean Squared error on $\hat{\boldsymbol{\beta}}_{OLS}$ with the strength of the correlations for various sample sizes and strength of regression. $d=5$ covariates (running example). } \label{MQEOLScompl}
	\end{figure}	
\end{frame}		
		\begin{frame}{Ridge}
		Alternative ridge : bien mais pas de sélection donc inacceptable
		\end{frame}
		
		\begin{frame}{Ridge}
		Alternative ridge : bien mais pas de sélection donc inacceptable
		\end{frame}
		
		\begin{frame}{LASSO}
		Alternative lasso (et autres) : bien mais problèmes en cas de corrélations
		\end{frame}
		\begin{frame}{LASSO}
		Liste des alternatives
		\end{frame}
		\begin{frame}{ SEM }
		 Modélisation de la structure mais à la main et aucun impact sur l'estimation
		 \end{frame}
		 
		\begin{frame}{ Selvarclust }
			Semble très bien mais n'aboutit pas vers la régression donc on le prolonge en CorReg
		\end{frame}
\section{Proposed Models}
	\begin{frame}{•}
	
	\end{frame}
	\subsection{Marginal model}
		\begin{frame}{•}
		
		\end{frame}
	\subsection{Plug-in model}
		\begin{frame}{•}
		
		\end{frame}
\section{Structure estimation}
	\begin{frame}{•}
	
	\end{frame}

\section{Results}
	\subsection{Simulation results}
		\begin{frame}{•}
		
		\end{frame}
	\subsection{Industrial results}
		\begin{frame}{•}
		
		\end{frame}
\section{Missing values}
	\begin{frame}{•}
	Modèle génératif complet avec dépendances
	\end{frame}
	\begin{frame}{•}
	Explosion des mélanges 
	\end{frame}
	\begin{frame}{•}
	SEM avec Gibbs 
	\end{frame}
	\begin{frame}{•}
	Bic pondéré 
	\end{frame}
	\begin{frame}{•}
	Résultats pourris 
	\end{frame}
\section{Tools}
	\begin{frame}{•}
		Excel, fonctions graphiques, arbres de décision
	\end{frame}

\begin{frame}
\bibliographystyle{apalike}
%\bibliographystyle{plain}
%\nocite{*}
\bibliography{biblio}
\end{frame}

\end{document}